\documentclass{article}
\usepackage[utf8]{inputenc}
\usepackage{geometry}
\usepackage{amsmath}
\usepackage{physics}
\usepackage{dsfont}
\usepackage{graphicx}

\let\vec\mathbf

\geometry{margin = 2.5cm}

\title{Analytical Approximations of Vacuum Magnetic Fields in Tokamaks}
\author{Alexander Prokopyshyn}
\date{\today}

\begin{document}

\maketitle

\section{Magnetic field produced by a vertical wire centred at the origin}

The magnetic field produce by a vertical wire, centred at the origin, with current, \(I\) which points in the 
\(\vec{\hat{h}}\) direction and extends from \(-h_{min}, h_{max}\) is given by the following Biot-Savart law expression:
\[\begin{aligned}
    \vec{B}_{v}(R, \phi, h) &= \frac{\mu_0 I}{4\pi}\int_{h_{min}}^{h_{max}}\frac{\vec{dh'}\cross(R\vec{\hat{R}} + (h-h')\vec{\hat{h}})}{\qty(R^2+(h-h')^2)^{3/2}} \\
    &= \frac{\mu_0 I}{4\pi}\int_{h_{min}}^{h_{max}}\frac{Rdh'\vec{\hat{\phi}}}{\qty(R^2+(h-h')^2)^{3/2}}.
\end{aligned}
\]
Let
\[\cos\theta = \frac{R}{\sqrt{R^2+(h-h')^2}},\]
\[\implies \sin\theta = \frac{h-h'}{\sqrt{R^2+(h-h')^2}},\]
\[\implies \tan \theta = \frac{h-h'}{R},\]
\[\implies h' = h - R\tan\theta,\]
\[\implies dh' = -R\sec^2\theta d\theta.\]
Hence
\[
\boxed{\begin{aligned}
    \vec{B}_{v}(R, \phi, h) &= -\frac{\mu_0 I}{4\pi R}\int_{h_{min}}^{h_{max}}\cos\theta d\theta\vec{\hat{\phi}}\\
    &= \frac{\mu_0 I}{4\pi R}\qty[-\sin\theta]_{h_{min}}^{h_{max}}\vec{\hat{\phi}}\\
    &= \frac{\mu_0 I}{4\pi R}\qty[\frac{h_{max}-h}{\sqrt{R^2+(h_{max}-h)^2}} + \frac{h-h_{min}}{\sqrt{R^2+(h-h_{min})^2}}]\vec{\hat{\phi}}.
\end{aligned}}\]

\section{Vacuum Field from Vertical Wires in Picture-Frame TF Coils}

To calculate the vacuum magnetic field induced by the vertical wires in the picture-frame toroidal
field coils of a Tokamak, it is convenient to express the field as a complex function, \( B_{v,c} \).
In this representation, the real and imaginary parts correspond to the \( x \)- and \( y \)-components
of the magnetic field, respectively. Furthermore, the radial and ahimuthal components can be obtained
from the real and imaginary parts of \( e^{-i\phi} B_{v,c} \), respectively.
Let
\[z = x + i y = R\exp(i\phi),\]
\[\bar{z} = x - iy = R \exp(-i\phi).\]
Hence,
\[B_{v,c}(z; h) = i\frac{\mu_0I}{4\pi}\frac{z}{\abs{z}^2}\qty[\frac{h_{max}-h}{\sqrt{|z|^2+(h_{max}-h)^2}} + \frac{h-h_{min}}{\sqrt{|z|^2+(h-h_{min})^2}}].\]
Let
\[z_k = R_0 e^{i\phi_k}.\]
Subsituting $z\rightarrow z- z_k$ above gives
\[B_{v,c}(z-z_k; h) = i\frac{\mu_0I}{4\pi}\frac{z-z_k}{\abs{z-z_k}^2}\qty[\frac{h_{max}-h}{\sqrt{|z-z_k|^2+(h_{max}-h)^2}} + \frac{h-h_{min}}{\sqrt{|z-z_k|^2+(h-h_{min})^2}}].\]
Let
\[\phi_k = 2\pi\frac{k}{N},\]
where $N$ equals the number of toroidal field (TF) coils. Hence, the magnetic field given a complete set of vertical wires
in the picture frame TF coils at $R=R_0$ is given by
\[B_{v,\Sigma,c}(z; h, R_0) = i\frac{\mu_0I}{4\pi}\sum_{k=0}^{N-1}\frac{z-z_k}{\abs{z-z_k}^2}\qty[\frac{h_{max}-h}{\sqrt{|z-z_k|^2+(h_{max}-h)^2}} + \frac{h-h_{min}}{\sqrt{|z-z_k|^2+(h-h_{min})^2}}].\]
At this stage it convenient to define a new function
\[B_{v,\Sigma,c,p}(z) = B_{v,\Sigma,c}(z)\exp(-i\phi),\]
where the $p$ stands for polar. Note that the real and imaginary components of $B_{v,\Sigma,c}$ give
the $x$ and $y$ components of the magnetic field, while the real and imaginary components of $B_{v,\Sigma,c,p}$
give the radial and azimuthal components of the magnetic field.

We now wish to express $B_{v,\Sigma,c,p}$ as a fourier series given by
\[B_{v,\Sigma,c,p}(z)=\sum_{n=-\infty}^\infty B_n(R) e^{in\phi}.\]
Note that
\[\begin{aligned}
B_n &= \frac{1}{2\pi}\int_0^{2\pi} B_{v,\Sigma,c,p}(z) e^{-in\phi}d\phi \\
&=\begin{cases}
    0 & \text{if } n \bmod N \ne 0, \\
    iN\frac{\mu_0I}{8\pi^2}\int_0^{2\pi}\frac{z-R_0}{\abs{z-R_0}^2}e^{-i\phi}\qty[\frac{h_{max}-h}{\sqrt{|z-R_0|^2+(h_{max}-h)^2}} + \frac{h-h_{min}}{\sqrt{|z-R_0|^2+(h-h_{min})^2}}]e^{-in\phi}d\phi & \text{if } n \bmod N = 0,
\end{cases}
\end{aligned}\]
where we have used the fact that $B_{v,\Sigma, c}(R, \phi) = B_{v,\Sigma, c}(R, \phi + \phi_k)$.
We now want to calculate this integral which is quite tricky so we will first consider the simpler scenario
where $h_{max},-h_{min}\rightarrow\infty$.

% We now wish to express $B_{v,\Sigma,c}$ as a fourier series given by
% \[B_{v,\Sigma,c}(z; h, R_0)=\sum_{n=-\infty}^\infty B_{n}(R; h, R_0) e^{in\phi}.\]
% Note that
% \[\begin{aligned}
% B_{n}(R; h, R_0) &= \frac{1}{2\pi}\int_0^{2\pi} B_{v,\Sigma,c}(z; h, R_0) e^{-in\phi}d\phi \\
% &=\begin{cases}
%     0 & \text{if } n \bmod N \ne 0, \\
%     iN\frac{\mu_0I}{8\pi^2}\int_0^{2\pi}\frac{z-R_0}{\abs{z-R_0}^2}\qty[\frac{h_{max}-h}{\sqrt{|z-R_0|^2+(h_{max}-h)^2}} + \frac{h-h_{min}}{\sqrt{|z-R_0|^2+(h-h_{min})^2}}]e^{-in\phi}d\phi & \text{if } n \bmod N = 0,
% \end{cases}
% \end{aligned}\]
% where we have used the fact that $B_{v,\Sigma, c}(R, \phi) = B_{v,\Sigma, c}(R, \phi + \phi_k)$.
% We now want to calculate this integral which is quite tricky so we will first consider the simpler scenario
% where $h_{max},-h_{min}\rightarrow\infty$.

\subsection{$B_{n}$ for case where $h_{max},-h_{min}\rightarrow\infty$}
\label{sec:case_where_h_max_infinite}

\[B_{n}(R) = iN\frac{\mu_0I}{4\pi^2}\int_0^{2\pi}\frac{z-R_0}{\abs{z-R_0}^2}e^{-i\phi}e^{-in\phi}d\phi.\]

Let
\[Z=e^{i\phi},\]
\[\implies d\phi = \frac{dZ}{iZ},\]
\[z=RZ.\]
Note that
\[\frac{z-R_0}{\abs{z-R_0}^2}e^{-i\phi} = \frac{1}{\bar{z} - R_0}e^{-i\phi} = \frac{1}{R-R_0e^{i\phi}}.\]
Hence
\[\begin{aligned}
B_{n}(R) &= N\frac{\mu_0I}{4\pi^2}\oint_{|Z|=1}\frac{1}{(R-R_0Z)Z^{n+1}}dZ \\
&= -N\frac{\mu_0I}{4\pi^2}\frac{1}{R_0}\oint_{|Z|=1}\frac{1}{(Z-R/R_0)Z^{n+1}}dZ
\end{aligned}\]
The $|Z|=1$ disk contains a pole of order $n$ at $Z=0$ if $n\ge 1$ and a pole of order 1 at
$Z=R/R_0$ if $R< R_0$.

For $n\le-1$ and $R< R_0$:
\[\begin{aligned}
B_{n}(R) &= -N\frac{\mu_0I}{4\pi^2}\frac{2\pi i}{R_0}\qty[Res(Z=R/R_0)] \\
&= -iN\frac{\mu_0I}{2\pi}\frac{1}{R_0}\qty(\frac{R}{R_0})^{-n-1}\\
&= -iN\frac{\mu_0I}{2\pi}\frac{1}{R}\qty(\frac{R}{R_0})^{-n}.
\end{aligned}\]

For $n\ge0$ and $R \le R_0$:
\[\begin{aligned}
B_{n}(R) &= -N\frac{\mu_0I}{4\pi^2}\frac{2\pi i}{R_0}\qty[Res(Z=R/R_0) + Res(Z=0)] \\
&= -N\frac{\mu_0I}{4\pi^2}\frac{2\pi i}{R_0}\qty[\qty(\frac{R}{R_0})^{-n} - \qty(\frac{R}{R_0})^{-n} ] \\
&= 0
\end{aligned}\]

For $n\le-1$ and $R> R_0$
\[B_{n}(R) = 0.\]

For $n\ge 0$ and $R> R_0$
\[\begin{aligned}
B_{n}(R) &= -N\frac{\mu_0I}{4\pi^2}\frac{2\pi i}{R_0}\qty[Res(Z=0)] \\
&= iN\frac{\mu_0I}{2\pi}\frac{1}{R}\qty(\frac{R_0}{R})^n.
\end{aligned}\]

Hence,
\[\boxed{B_{v,\Sigma,c,p} = i\frac{\mu_0NI}{2\pi R}\begin{cases}
-\qty(R/R_0)^Ne^{-iN\phi} - \qty(R/R_0)^{2N}e^{-2iN\phi} - ... & \text{if } R<R_0\\
1 + \qty(R_0/R)^Ne^{iN\phi} + \qty(R_0/R)^{2N}e^{2iN\phi} + ... & \text{if } R>R_0.\\
\end{cases}}\]
Note that $B_R$ and $B_\phi$ are given by the real and imaginary parts of the above expression. Hence
\[\boxed{B_R = \frac{\mu_0NI}{2\pi R}\begin{cases}
-\qty(R/R_0)^N\sin(N\phi) - \qty(R/R_0)^{2N}\sin(2N\phi) - ... & \text{if } R<R_0\\
1 + \qty(R_0/R)^N\sin(N\phi) + \qty(R_0/R)^{2N}\sin(2N\phi) + ... & \text{if } R>R_0,\\
\end{cases}}\]
\[\boxed{B_\phi = \frac{\mu_0NI}{2\pi R}\begin{cases}
-\qty(R/R_0)^N\cos(N\phi) - \qty(R/R_0)^{2N}\cos(2N\phi) - ... & \text{if } R<R_0\\
1 + \qty(R_0/R)^N\cos(N\phi) + \qty(R_0/R)^{2N}\cos(2N\phi) + ... & \text{if } R>R_0.\\
\end{cases}}\]


\subsection{$B_{n}$ for case where $2RR_0/[(R^2+R_0^2+a^2)]\ll1$}

Consider the function
\[\begin{aligned}
f_a(z) &= \frac{a}{\sqrt{\abs{z-R_0}^2+a^2}} \\
&= \frac{a}{\sqrt{R^2+R_0^2 + 2RR_0\cos\phi+a^2}}.
\end{aligned}\]
By the AM-GM inequality
\[\frac{2RR_0}{R^2+R_0^2}\le 1,\]
with its maximum value at $R=R_0$.
Let
\[\epsilon=\frac{2RR_0}{(R^2+R_0^2+a^2)}\]
We can approximate $f_a(z)$ with the following binomial expansion
\[\begin{aligned}
f_a(z) &= \frac{a}{\sqrt{R^2+R_0^2+a^2}\sqrt{1 + \epsilon\cos\phi}} \\
&= \frac{a}{\sqrt{R^2+R_0^2+a^2}}\qty(1 - \frac{1}{2}\epsilon\cos\phi
+ O(\epsilon^2)).
\end{aligned}\]
Hence, using the equation at the end of Section \ref{sec:case_where_h_max_infinite} and 
substituting $a\rightarrow h_{max} - h$ and $a\rightarrow h - h_{max}$ gives
\[\boxed{B_R(R, \phi, h) = \frac{\mu_0NI}{4\pi R}\begin{cases}
-\qty(\frac{R}{R_0})^N\qty(\frac{h_{max}-h}{\sqrt{R^2+R_0^2+(h_{max}-h)^2}}+\frac{h - h_{min}}{\sqrt{R^2+R_0^2+(h-h_{min})^2}}+ O(\epsilon))\sin(N\phi) +O(\frac{R}{R_0})^{2N} & \text{if } R<R_0\\
1 + \qty(\frac{R}{R_0})^N\qty(\frac{h_{max}-h}{\sqrt{R^2+R_0^2+(h_{max}-h)^2}}+\frac{h - h_{min}}{\sqrt{R^2+R_0^2+(h-h_{min})^2}}+ O(\epsilon))\sin(N\phi) +O(\frac{R}{R_0})^{2N} & \text{if } R>R_0,\\
\end{cases}}\]
\[\boxed{B_Z(R, \phi, h) = \frac{\mu_0NI}{4\pi R}\begin{cases}
1 + \qty(\frac{R}{R_0})^N\qty(\frac{h_{max}-h}{\sqrt{R^2+R_0^2+(h_{max}-h)^2}}+\frac{h - h_{min}}{\sqrt{R^2+R_0^2+(h-h_{min})^2}}+ O(\epsilon))\cos(N\phi) +O(\frac{R}{R_0})^{2N} & \text{if } R>R_0,\\
\end{cases}}\]


% \[
% Res(Z=0) = \begin{cases}
%     \frac{1}{R_0}\qty(\frac{R}{R_0})^n & \text{if n\ge 1}
% \end{cases}\]

% Let
% \[I_n(R) = \int_0^{2\pi}\frac{z-R_0}{\abs{z-R_0}^2}\qty[\frac{h_{max}-h}{\sqrt{|z-R_0|^2+(h_{max}-h)^2}} + \frac{h-h_{min}}{\sqrt{|z-R_0|^2+(h-h_{min})^2}}]e^{-in\phi}d\phi.\]
% Let
% \[z' = e^{i\phi}.\]
% \[\implies d\phi = \frac{dz'}{iz'}.\]
% Note that
% \[\frac{z-R_0}{\abs{z-R_0}^2} = \frac{1}{\bar{z} - R_0}=\frac{1}{R\exp(-i\phi) - R_0}= \frac{z'}{R-R_0z'},\]
% \[\abs{z-R_0}^2 = R^2 + R_0^2 - R_0R(e^{i\phi} + e^{-i\phi})= R^2 + R_0^2 - R_0R(z' + 1/z').\]
% Let
% \[f(z') = \frac{a}{\sqrt{R^2+R_0^2 - R_0R(z'+1/z')}}\]
% \[\frac{a}{\sqrt{R^2+R_0^2 - R_0R(z'+1/z')}}=\]
% Hence,
% \[I_n(R) = \frac{1}{i}\oint_{|z'|=1}\frac{(z')^{-n}}{R - R_0z'}\qty[\frac{h_{max}-h}{\sqrt{R^2+R_0^2-RR_0(z'+1/z')+(h_{max}-h)^2}} + \frac{h-h_{min}}{\sqrt{R^2+R_0^2-RR_0(z'+1/z')+(h-h_{min})^2}}]dz' \]
% $$I=\oint_{|z=1|} \frac{z^{-n}}{R-R_0z}\frac{a}{\sqrt{R^2+R_0^2+a^2-RR_0(z+1/z)}}dz$$
\end{document}
