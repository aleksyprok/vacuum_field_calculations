\documentclass{article}
\usepackage[utf8]{inputenc}
\usepackage{geometry}
\usepackage{amsmath}
\usepackage{physics}
\usepackage{dsfont}
\usepackage{graphicx}
\usepackage{hyperref}

\let\vec\mathbf

\geometry{margin = 2.5cm}

\title{Analytic Approximation of the Vacuum Magnetic Field Induced by Infinitely Wide Picture Frame coils}
\author{Alexander Prokopyshyn}
\date{\today}

\begin{document}

\maketitle

\section{Magnetic field induced by a vertical wire centred at the origin}
\label{sec:vertical_wire_at_origin}

The magnetic field induced by a vertical wire, centred at the origin, with current, \(I_0\) which points in the 
\(\vec{\hat{z}}\) direction and extends from \(-z_{min}, z_{max}\) is given by the following Biot-Savart law expression:
\[\begin{aligned}
    \vec{B}_{v}(R, \phi, z) &= \frac{\mu_0 I_0}{4\pi}\int_{z_{min}}^{z_{max}}\frac{\vec{dl'}\cross(R\vec{\hat{R}} + (z-z')\vec{\hat{z}})}{\qty(R^2+(z-z')^2)^{3/2}} \\
    &= \frac{\mu_0 I_0}{4\pi}\int_{z_{min}}^{z_{max}}\frac{Rdh'\vec{\hat{\phi}}}{\qty(R^2+(z-z')^2)^{3/2}}.
\end{aligned}
\]
Let
\[\cos\theta = \frac{R}{\sqrt{R^2+(z-z')^2}},\]
\[\implies \sin\theta = \frac{z-z'}{\sqrt{R^2+(z-z')^2}},\]
\[\implies \tan \theta = \frac{z-z'}{R},\]
\[\implies z' = z - R\tan\theta,\]
\[\implies dz' = -R\sec^2\theta d\theta.\]
Hence
\[
\begin{aligned}
    \vec{B}_{v}(R, \phi, z) &= -\frac{\mu_0 I_0}{4\pi R}\int_{z_{min}}^{z_{max}}\cos\theta d\theta\vec{\hat{\phi}}\\
    &= \frac{\mu_0 I_0}{4\pi R}\qty[-\sin\theta]_{z_{min}}^{z_{max}}\vec{\hat{\phi}}\\
    &= \frac{\mu_0 I_0}{4\pi R}\qty[\frac{z_{max}-z}{\sqrt{R^2+(z_{max}-z)^2}} + \frac{z-z_{min}}{\sqrt{R^2+(z-z_{min})^2}}]\vec{\hat{\phi}}.
\end{aligned}\]
Let $z_{max}=h$ and $z_{min}=-h$:
\[
\boxed{\begin{aligned}
    \vec{B}_{v}(R, \phi, z) &= -\frac{\mu_0 I_0}{4\pi R}\int_{z_{min}}^{z_{max}}\cos\theta d\theta\vec{\hat{\phi}}\\
    &= \frac{\mu_0 I_0}{4\pi R}\qty[-\sin\theta]_{z_{min}}^{z_{max}}\vec{\hat{\phi}}\\
    &= \frac{\mu_0 I_0}{4\pi R}\qty[\frac{h-z}{\sqrt{R^2+(h-z)^2}} + \frac{z+h}{\sqrt{R^2+(z+h)^2}}]\vec{\hat{\phi}}.
\end{aligned}}\]

\section{Magnetic field induced by a horizontal wire extending from the origin to infinity}

Let $\vec{B}_{h,c, k}(x,y,z)$ denote the magnetic field induced by a horizontal
wire in cartesian coordinates extending from the origin to infinity with current, $I_0$, in the
\[\vec{\hat{R}}_k=\cos(\phi_k)\vec{\hat{x}} + \sin(\phi_k)\vec{\vec{\hat{y}}}\]
direction, where
\[\phi_k = k\frac{2\pi}{N},\]
$N$ is the number of picture-frame coils. For $k=0$, $\vec{B}_{h,c,k}(x,y,z)$ points in the $\vec{\hat{x}}$ direction
and is given by
\[\begin{aligned}
    \vec{B}_{h,c,0}(x, y, z) &= \frac{\mu_0 I_0}{4\pi}\frac{-z\vec{\hat{y}} + y\vec{\hat{z}}}{y^2+z^2}\left[1+\frac{x}{\sqrt{x^2+y^2+z^2}}.\right]
\end{aligned}\]
Let $\vec{B}_{h,k}$ denote the same magnetic field as $\vec{B}_{h,c,k}$ but in cylindrical coordinates. Then
\[\begin{aligned}
    \vec{B}_{h,0}(R, \phi, z) &= \frac{\mu_0 I_0}{4\pi}\frac{-z\sin\phi\vec{\hat{R}} -z\cos\phi\vec{\hat{\phi}} + R\sin\phi\vec{\hat{z}}}{R^2\sin^2\phi+z^2}\left[1+\frac{R\cos\phi}{\sqrt{R^2+z^2}}\right].
\end{aligned}\]

\section{Fourier series of the magnetic field induced by a horizontal wires extending from the origin to infinity}

Let 
\[\vec{B}_{h}(R,\phi,z)=\sum_{k=0}^{N-1}\vec{B}_{h,k}(R, \phi, z).\]
We now wish to express
\[\vec{B}_{h}(R,\phi,z)=B_{h, R}(R, \phi, z)\vec{\hat{R}} + B_{h, \phi}(R, \phi, z)\vec{\hat{\phi}} + B_{h, z}(R, \phi, z)\vec{\hat{z}},\]
as a Fourier series in $\phi$. The picture-frame coils are positioned to ensure that $B_{h, R}$ and $B_{h, z}$ are odd
functions of $\phi$ and $B_{h, \phi}$ is an even function of $\phi$. Hence
\[B_{h,R}(R, \phi, z) = \sum_{n=N}^{\infty}B_{h, R, n}(R, z)\sin(n\phi),\]
\[B_{h,\phi}(R, \phi, z) = \sum_{n=0}^{\infty}B_{h, \phi, n}(R, z)\cos(n\phi),\]
\[B_{h, z}(R, \phi, z) = \sum_{n=N}^{\infty}B_{h, z, n}(R, z)\sin(n\phi).\]
where $B_{h, R, n}(R, z)=B_{h, \phi, n}(R, z)=B_{h, z, n}(R, z)=0$ for $n \bmod N \ne 0$.
To be clear, $B_{h, R, n}(R, z)\ne B_{h, R, n}(R, \phi, z)$, $B_{h, \phi, n}(R, z)\ne B_{h, \phi, n}(R, \phi, z)$,
$B_{h, z, n}(R, z)\ne B_{h, z, n}(R, \phi, z)$. From this point on we will assume $n \bmod N = 0$.

Note that
\[\begin{aligned}
B_{h, R, n}(R, z)&=\frac{1}{\pi}\int_{0}^{2\pi}B_{h, R}(R, \phi, z)\sin(n\phi)d\phi \\
&= \frac{N}{\pi}\int_0^{2\pi} B_{h, R, 0}(R, \phi, z)\sin(n\phi)d\phi,
\end{aligned}\]
for $n \bmod N = 0$.
\[\begin{aligned}
B_{h, \phi, 0}(R, z)&=\frac{1}{2\pi}\int_{0}^{2\pi}B_{h, \phi}(R, \phi, z)d\phi \\
&= \frac{N}{2\pi}\int_0^{2\pi} B_{h, \phi, 0}(R, \phi, z)d\phi,
\end{aligned}\]
\[\begin{aligned}
B_{h, \phi, n}(R, z)&=\frac{1}{\pi}\int_{0}^{2\pi}B_{h, \phi}(R, \phi, z)\cos(n\phi)d\phi \\
&= \frac{N}{\pi}\int_0^{2\pi} B_{h, \phi, 0}(R, \phi, z)\cos(n\phi)d\phi,
\end{aligned}\]
for $n \bmod N = 0$ and $n\ne0$.
\[\begin{aligned}
B_{h, z, n}(R, z)&=\frac{1}{\pi}\int_{0}^{2\pi}B_{h, z}(R, \phi, z)\sin(n\phi)d\phi \\
&= \frac{N}{\pi}\int_0^{2\pi} B_{h, z, 0}(R, \phi, z)\sin(n\phi)d\phi,
\end{aligned}\]
for $n \bmod N = 0$.

\subsection{Fourier coefficients of the radial component of the magnetic field}

Our goal in this section is to calculate this integral
\[\begin{aligned}
B_{R, h, n}(R, z) &= \frac{N}{\pi}\int_0^{2\pi} B_{R, h, 0}(R, \phi, z)\sin(n\phi)d\phi \\
&= -\frac{N}{\pi}\frac{\mu_0 I_0}{4 \pi}\int_0^{2\pi} \frac{z\sin\phi}{R^2\sin^2\phi+z^2}\qty(1+\frac{R\cos\phi}{\sqrt{R^2+z^2}})\sin(n\phi)d\phi \\
&= -\frac{N}{\pi}\frac{\mu_0 I_0}{4 \pi}\frac{1}{R}\int_0^{2\pi} \frac{\Delta\sin\phi}{\sin^2\phi+\Delta^2}\qty(1+\frac{\cos\phi}{\sqrt{1+\Delta^2}})\sin(n\phi)d\phi.
\end{aligned}\]
Note that
\[\int_{0}^{2\pi}\frac{\sin(\phi)\sin(n\phi)}{\sin^2\phi+\Delta^2}d\phi = 0,\]
for $n$ odd.
\[\begin{aligned}
\int_{0}^{2\pi}\frac{\sin(\phi)\cos(\phi)\sin(n\phi)}{\sin^2\phi+\Delta^2}d\phi &= \frac{1}{2}\int_{0}^{2\pi}\frac{\sin(2\phi)\sin(n\phi)}{\sin^2\phi+\Delta^2}d\phi \\
&=\int_{0}^{2\pi}\frac{\sin(2\phi)\sin(n\phi)}{1+2\Delta^2-\cos(2\phi)}d\phi. \\
\end{aligned}\]
Let
\[\Delta^2 = \frac{(1-a)^2}{4a},\]
\[a = \qty(\sqrt{1+\Delta^2}-\abs{\Delta})^2.\]
Hence,
\[\begin{aligned}
\int_{0}^{2\pi}\frac{\sin(\phi)\cos(\phi)\sin(n\phi)}{\sin^2\phi+\Delta^2}d\phi &= 2\int_{0}^{2\pi}\frac{a\sin(2\phi)}{1+a^2-2a\cos(2\phi)}\sin(n\phi)d\phi. \\
\end{aligned}\]
Note that
\[\begin{aligned}
\frac{1}{1-a\exp(2i\phi)} &= \frac{1-a\cos(2\phi)}{1+a^2-2a\cos(2\phi)}+i\frac{a\sin(2\phi)}{1+a^2-2a\cos(2\phi)} \\
&= \sum_{k=0}^{\infty}a^k\exp(2ik\phi).
\end{aligned}\]
Taking the imaginary part gives
\[\frac{a\sin(2\phi)}{1+a^2-2a\cos(2\phi)}=\sum_{k=1}a^k\sin(2k\phi).\]
Hence,
\[\begin{aligned}
\int_{0}^{2\pi}\frac{\sin(\phi)\cos(\phi)\sin(n\phi)}{\sin^2\phi+\Delta^2}d\phi &= 2\pi a^{n/2} \\
&= 2\pi\qty(\sqrt{1+\Delta^2}-\abs{\Delta})^n
\end{aligned}\]

Therefore,
\[\begin{aligned}
B_{R, h, n}(R, z) &= -\frac{N}{\pi}\frac{\mu_0 I_0}{4 \pi}\frac{1}{R}\int_0^{2\pi} \frac{\Delta\sin\phi}{\sin^2\phi+\Delta^2}\qty(1+\frac{\cos\phi}{\sqrt{1+\Delta^2}})\sin(n\phi)d\phi \\
&= -\frac{N}{\pi}\frac{\mu_0 I_0}{4 \pi}\frac{1}{R}\frac{\Delta}{\sqrt{1+\Delta^2}}2\pi\qty(\sqrt{1+\Delta^2}-\abs{\Delta})^n \\
&= -N\frac{\mu_0 I_0}{2 \pi}\frac{1}{R}\frac{z}{\sqrt{R^2+z^2}}\qty(\frac{\sqrt{R^2+z^2}-\abs{z}}{R})^n \\
\end{aligned}\]

% Let
% \[I_1 = \Delta\int_{0}^{2\pi}\frac{\sin\phi\sin (n\phi)}{\sin^2\phi+\Delta^2}d\phi,\]
% \[I_2 = \frac{\Delta}{2\sqrt{1+\Delta^2}}\int_{0}^{2\pi}\frac{\sin(2\phi)\sin (n\phi)}{\sin^2\phi+\Delta^2}d\phi.\]
% To solve $I_1$ and $I_2$ we use a trick where we use a trick where we let $Z=e^{i\phi}$ and substitute
% \[\sin(n\phi) = \frac{Z^n+Z^{-n}}{2i}.\]
% Then we take the integral over the unit circle $\abs{Z}=1$.

\subsection{Fourier coefficients of the azimuthal component of the magnetic field}

Our goal in this section is to calculate these integrals
\[\begin{aligned}
B_{h, \phi, 0}(R, z)&=\frac{1}{2\pi}\int_{0}^{2\pi}B_{h, \phi}(R, \phi, z)d\phi \\
&= -\frac{N}{2\pi}\frac{\mu_0 I_0}{4 \pi}\int_0^{2\pi} \frac{z\cos\phi}{R^2\sin^2\phi+z^2}\qty(1+\frac{R\cos\phi}{\sqrt{R^2+z^2}})d\phi \\
&= -\frac{N}{2\pi}\frac{\mu_0 I_0}{4 \pi}\frac{1}{R}\int_0^{2\pi} \frac{\Delta\cos\phi}{\sin^2\phi+\Delta^2}\qty(1+\frac{\cos\phi}{\sqrt{1+\Delta^2}})d\phi,
\end{aligned}\]
\[\begin{aligned}
B_{h, \phi, n}(R, z)&=\frac{1}{\pi}\int_{0}^{2\pi}B_{h, \phi}(R, \phi, z)\cos(n\phi)d\phi \\
&= -\frac{N}{\pi}\frac{\mu_0 I_0}{4 \pi}\int_0^{2\pi} \frac{z\cos\phi}{R^2\sin^2\phi+z^2}\qty(1+\frac{R\cos\phi}{\sqrt{R^2+z^2}})\cos(n\phi)d\phi \\
&= -\frac{N}{\pi}\frac{\mu_0 I_0}{4 \pi}\frac{1}{R}\int_0^{2\pi} \frac{\Delta\cos\phi}{\sin^2\phi+\Delta^2}\qty(1+\frac{\cos\phi}{\sqrt{1+\Delta^2}})\cos(n\phi)d\phi,
\end{aligned}\]
for $n \bmod N = 0$ and $n\ne0$,
where
\[\Delta = z/R.\]
Note that,
\[\int_0^{2\pi} \frac{\cos\phi}{\sin^2\phi+\Delta^2}d\phi = 0,\]
\[\int_0^{2\pi} \frac{\cos^2\phi}{\sin^2\phi+\Delta^2}d\phi = 2\pi\qty(\frac{\sqrt{1+\Delta^2}}{|\Delta|}-1).\]
Hence,
\[\begin{aligned}
    B_{h, \phi, 0}(R, z)&=-\frac{N}{2\pi}\frac{\mu_0 I_0}{4 \pi}\frac{1}{R} \frac{\Delta}{\sqrt{1+\Delta^2}} 2\pi\qty(\frac{\sqrt{1+\Delta^2}}{|\Delta|}-1) \\
    &= N\frac{\mu_0 I_0}{4 \pi}\frac{1}{R}\qty(\frac{z}{\sqrt{R^2+z^2}}-\frac{z}{|z|}).
\end{aligned}
\]

Note that
\[\int_0^{2\pi} \frac{\cos\phi}{\sin^2\phi+\Delta^2}\cos(n\phi)d\phi = 0,\]
for $n$ even.
\[\begin{aligned}
\int_0^{2\pi} \frac{\cos^2\phi}{\sin^2\phi+\Delta^2}\cos(n\phi)d\phi &= \frac{2\pi}{|\Delta|}\sqrt{1 + \Delta^2}\qty(\sqrt{1+\Delta^2}-\abs{\Delta})^n,
\end{aligned}\]
for $n\ge 2$ and even. For proof, see \href{https://math.stackexchange.com/questions/5027889/how-to-express-int-pi-pi-frac-cos2x-cosnx-sin2xa2dx-as-a#5027947}{this stack exchange link}.
Hence,
\[\begin{aligned}
    B_{h, \phi, n}(R, z)&=-\frac{N}{\pi}\frac{\mu_0 I_0}{4 \pi}\frac{1}{R} \frac{\Delta}{\sqrt{1+\Delta^2}} \frac{2\pi}{|\Delta|}\sqrt{1 + \Delta^2}\qty(\sqrt{1+\Delta^2}-\abs{\Delta})^n \\
    &= -\frac{z}{\abs{z}}N\frac{\mu_0 I_0}{2 \pi}\frac{1}{R}\qty(\sqrt{1+\qty(\frac{z}{R})^2}-\abs{\frac{z}{R}})^n \\
    &= -\frac{z}{\abs{z}}N\frac{\mu_0 I_0}{2 \pi}\frac{1}{R}\qty(\frac{\sqrt{R^2+z^2}-\abs{z}}{R})^n.
\end{aligned}
\]

% Solutions to 
% \[\int_0^{2\pi} \frac{\cos^2\phi}{\sin^2\phi+\Delta^2}\cos(n\phi)d\phi\]
% are quite complicated and so we will take an asymptotic approach.
% Note that
% \[\begin{aligned}
% \int_0^{2\pi} \frac{\cos^2\phi}{\sin^2\phi+\Delta^2}\cos(n\phi)d\phi &=R^2\int_0^{2\pi} \frac{\cos^2\phi\cos(n\phi)}{R^2+z^2-R^2\cos^2\phi}d\phi\\
% &= \epsilon^2\int_0^{2\pi} \frac{\cos^2\phi\cos(n\phi)}{1-\epsilon^2\cos^2\phi}d\phi \\
% &= \epsilon^2\qty(\frac{\pi}{2}\qty(\frac{\epsilon}{2} )^{n-2}+ O[\epsilon^{n}])
% \end{aligned}\]

\subsection{Fourier coefficients of the vertical component of the magnetic field}

Our goal in this section is to calculate this integral
\[\begin{aligned}
B_{h, z, n}(R, z)&= \frac{N}{\pi}\int_0^{2\pi} B_{h, z, 0}(R, \phi, z)\sin(n\phi)d\phi \\
&=\frac{N}{\pi}\frac{\mu_0 I_0}{4 \pi}\int_0^{2\pi} \frac{R\sin\phi}{R^2\sin^2\phi+z^2}\qty(1+\frac{R\cos\phi}{\sqrt{R^2+z^2}})\sin(n\phi)d\phi \\
&= -\frac{R}{z}B_{h, R, n}(R, z)
\end{aligned}\]
for $n \bmod N = 0$.
Since
\[\begin{aligned}
B_{R, h, n}(R, z) &= -N\frac{\mu_0 I_0}{2 \pi}\frac{1}{R}\frac{z}{\sqrt{R^2+z^2}}\qty(\frac{\sqrt{R^2+z^2}-\abs{z}}{R})^n \\
\end{aligned}\]
we know that
\[\begin{aligned}
B_{z, h, n}(R, z) &= N\frac{\mu_0 I_0}{2 \pi}\frac{1}{R}\frac{R}{\sqrt{R^2+z^2}}\qty(\frac{\sqrt{R^2+z^2}-\abs{z}}{R})^n \\
\end{aligned}\]

\section{Full magnetic field}

The full magnetic field is given by
\[\mathbf{B}(R,\phi,z) = N\vec{B}_{v}(R, \phi, z) + \vec{B}_h(R, \phi, z-h) - \vec{B}_h(R, \phi, z+h).\]
Hence,
\[\boxed{
\begin{aligned}
B_R(R,\phi,z) = N\frac{\mu_0 I_0}{2 \pi}\frac{1}{R}\sum_{n=N}^{\infty}&\left[\frac{h-z}{\sqrt{R^2+(z-h)^2}}\qty(\frac{\sqrt{R^2+(z-h)^2}-\abs{z-h}}{R})^n+\right. \\
&\ \ \left.\frac{z+h}{\sqrt{R^2+(z+h)^2}}\qty(\frac{\sqrt{R^2+(z+h)^2}-\abs{z+h}}{R})^n\right]\sin(n\phi),
\end{aligned}
}\]
\[\boxed{
\begin{aligned}
B_\phi(R,\phi,z) = N\frac{\mu_0 I_0}{2 \pi}\frac{1}{R}\Bigg\{1+\sum_{n=N}^{\infty}&\left[\qty(\frac{\sqrt{R^2+(z-h)^2}-\abs{z-h}}{R})^n+\right. \\
&\ \ \left.\qty(\frac{\sqrt{R^2+(z+h)^2}-\abs{z+h}}{R})^n\right]\Bigg\}\cos(n\phi),
\end{aligned}
}\]
\[\boxed{
\begin{aligned}
B_z(R,\phi,z) = N\frac{\mu_0 I_0}{2 \pi}\frac{1}{R}\sum_{n=N}^{\infty}&\left[\frac{R}{\sqrt{R^2+(z-h)^2}}\qty(\frac{\sqrt{R^2+(z-h)^2}-\abs{z-h}}{R})^n-\right. \\
&\ \ \left.\frac{R}{\sqrt{R^2+(z+h)^2}}\qty(\frac{\sqrt{R^2+(z+h)^2}-\abs{z+h}}{R})^n\right]\sin(n\phi),
\end{aligned}
}\]
for $n \bmod N =0$ and $-h<z<h$.

\section{Check divergence is zero}

\[\div{\vec{B}}=\frac{1}{R}\frac{\partial}{\partial R}(RB_R) + \frac{1}{R}\frac{\partial B_\phi}{\partial \phi} + \pdv{B_z}{z}.\]
\[\begin{aligned}
\frac{1}{R}\frac{\partial}{\partial R}(RB_{R,h,n}(R,z)) &= \frac{1}{R}\frac{\partial}{\partial R}\qty{-N\frac{\mu_0 I_0}{2 \pi}\frac{z}{\sqrt{R^2+z^2}}\qty(\frac{\sqrt{R^2+z^2}-\abs{z}}{R})^n} \\
&= -N\frac{\mu_0 I_0}{2 \pi}\frac{z}{R}\frac{\partial}{\partial R}\qty{\frac{1}{\sqrt{R^2+z^2}}\qty(\frac{\sqrt{R^2+z^2}-\abs{z}}{R})^n} \\
&= -N\frac{\mu_0 I_0}{2 \pi}\frac{z}{R}\left\{-\frac{R}{(R^2+z^2)^{3/2}}\qty(\frac{\sqrt{R^2+z^2}-\abs{z}}{R})^n + \right. \\
&\qquad\qquad\qquad\qquad \left.n\frac{|z|\sqrt{R^2+z^2}-z^2}{R^2(R^2+z^2)}\qty(\frac{\sqrt{R^2+z^2}-\abs{z}}{R})^{n-1}\right\} \\
&= -N\frac{\mu_0 I_0}{2 \pi}\frac{z}{R}\left\{-\frac{R}{(R^2+z^2)^{3/2}}\qty(\frac{\sqrt{R^2+z^2}-\abs{z}}{R})^n + \right. \\
&\qquad\qquad\qquad\qquad \left.n\frac{|z|}{R(R^2+z^2)}\qty(\frac{\sqrt{R^2+z^2}-\abs{z}}{R})^{n}\right\} \\
&= -N\frac{\mu_0 I_0}{2 \pi}\frac{1}{R}\left\{-\frac{zR}{(R^2+z^2)^{3/2}}\qty(\frac{\sqrt{R^2+z^2}-\abs{z}}{R})^n + \right. \\
&\qquad\qquad\qquad\qquad \left.n\frac{|z|}{z}\frac{z^2}{R(R^2+z^2)}\qty(\frac{\sqrt{R^2+z^2}-\abs{z}}{R})^{n}\right\} \\
\end{aligned}\]
\[\begin{aligned}
    \pdv{B_{z,h,n}(R,z)}{z}&=N\frac{\mu_0 I_0}{2 \pi}\frac{\partial}{\partial z}\qty{\frac{1}{\sqrt{R^2+z^2}}\qty(\frac{\sqrt{R^2+z^2}-\abs{z}}{R})^n} \\ 
    &= N\frac{\mu_0 I_0}{2 \pi}\left\{-\frac{z}{(R^2+z^2)^{3/2}}\qty(\frac{\sqrt{R^2+z^2}-\abs{z}}{R})^n+\right. \\
    &\qquad\qquad\qquad\left.n\frac{z|z|-z\sqrt{R^2+z^2}}{R|z|(R^2+z^2)}\qty(\frac{\sqrt{R^2+z^2}-\abs{z}}{R})^{n-1}\right\} \\
    &= N\frac{\mu_0 I_0}{2 \pi}\left\{-\frac{z}{(R^2+z^2)^{3/2}}\qty(\frac{\sqrt{R^2+z^2}-\abs{z}}{R})^n-\right. \\
    &\qquad\qquad\qquad\left.n\frac{z}{|z|(R^2+z^2)}\qty(\frac{\sqrt{R^2+z^2}-\abs{z}}{R})^{n}\right\} \\
    &= N\frac{\mu_0 I_0}{2 \pi}\frac{1}{R}\left\{-\frac{zR}{(R^2+z^2)^{3/2}}\qty(\frac{\sqrt{R^2+z^2}-\abs{z}}{R})^n-\right. \\
    &\qquad\qquad\qquad\left.n\frac{z}{|z|}\frac{R^2}{R(R^2+z^2)}\qty(\frac{\sqrt{R^2+z^2}-\abs{z}}{R})^{n}\right\}.
\end{aligned}\]
Hence,
\[\begin{aligned}
    \frac{1}{R}\frac{\partial}{\partial R}(RB_{R,h,n}) + \pdv{B_{z,h,n}}{z} = -n\frac{z}{|z|}N\frac{\mu_0 I_0}{2 \pi}\frac{1}{R^2}\qty(\frac{\sqrt{R^2+z^2}-\abs{z}}{R})^{n}.
\end{aligned}\]
\[\implies \frac{1}{R}\frac{\partial}{\partial R}(RB_{R,h,n}) -\frac{n}{R}B_{\phi,h,n} + \pdv{B_{z,h,n}}{z} = 0,\]
which confirms that the divergence of the magnetic field is zero.

\end{document}